% https://yuxtech.club/tex/lshort-zh-cn.pdf
% https://tug.ctan.org/macros/latex/contrib/beamer/doc/beameruserguide.pdf
% \documentclass[⟨options⟩]{⟨class-name⟩}
\documentclass{beamer} % 幻灯格式的文档类,使用无衬线字体。

%% 字体
% ctex – LATEX classes and packages for Chinese typesetting
% https://ftp.jaist.ac.jp/pub/CTAN/language/chinese/ctex/ctex.pdf
%\usepackage[⟨options⟩]{⟨package-name⟩}
\usepackage{ctex} % 添加中文及版式的支持

%% 字体
% fontenc – Standard package for selecting font encodings
\usepackage[T1]{fontenc} % 切换字体编码

%% 字体 | 颜色
% xcolor – Driver-independent color extensions for LATEX and pdfLATEX
% https://ftp.yz.yamagata-u.ac.jp/pub/CTAN/macros/latex/contrib/xcolor/xcolor.pdf
\usepackage{xcolor} % 切换字体颜色

%% 段落 | 代码
% listings – Typeset source code listings using LATEX
% https://ftp.jaist.ac.jp/pub/CTAN/macros/latex/contrib/listings/listings.pdf
\usepackage{listings} % 提供了排版关键字高亮的代码环境 lstlisting 以及对版式的自定义。类似宏包有 minted



%% 表格
% booktabs – Publication quality tables in LATEX
\usepackage{booktabs} % 排版三线表

%% 插图 | 图片
% graphicx – Enhanced support for graphics
\usepackage{graphicx} % 支持插图

%% 插图
% pgf – Create PostScript and PDF graphics in TEX
\usepackage{tikz} % 支持绘图


%% 符号 | 公式
% amsmath – AMS mathematical facilities for LATEX
% https://ftp.yz.yamagata-u.ac.jp/pub/CTAN/macros/latex/required/amsmath/amsldoc.pdf
\usepackage{amsmath} % AMS 数学公式扩展

%% 符号 | 公式
% siunitx – A comprehensive (SI) units package
% https://ftp.yz.yamagata-u.ac.jp/pub/CTAN/macros/latex/contrib/siunitx/siunitx.pdf
\usepackage{siunitx} % 单位符号

%% 符号 | 符号
% latexsym - The LaTeX symbol fonts for use with LaTeX2ε.
% http://mirrors.ibiblio.org/CTAN/macros/latex/base/latexsym.pdf
\usepackage{latexsym} % 带有角标 ℓ 的符号命令依赖 latexsym 宏包。

%% 布局 | 栏
% multicol – Intermix single and multiple columns
% https://ftp.yz.yamagata-u.ac.jp/pub/CTAN/macros/latex/required/tools/multicol.pdf
\usepackage{multicol} % 提供将内容自由分栏的 multicols 环境


%% 引用
% BibLATEX – Sophisticated Bibliographies in LATEX
% https://ftp.kddilabs.jp/CTAN/macros/latex/contrib/biblatex/doc/biblatex.pdf
% backend=bibtex, bibtex8, biber default: biber
% Specifies the database backend. The following backends are supported:
\usepackage[backend=bibtex,sorting=none]{biblatex}
\addbibresource{ref.bib} % 为 biblatex 引入参考文献数据库
\defbibheading{references}{\section*{参考文献}}   % heading for bibtex

%% 引用
% hyperref – Extensive support for hypertext in LATEX
% https://ftp.yz.yamagata-u.ac.jp/pub/CTAN/macros/latex/contrib/hyperref/doc/hyperref-doc.pdf
% 为减少可能的冲突,习惯上将 hyperref 宏包放在其它宏包之后调用。
\usepackage{hyperref} % hyperref 宏包涉及到的链接遍布 LATEX 的每一个角落——目录、引用、脚注、索引、参考文献等等都被封装成超链接。

\setbeamerfont{footnote}{size=\tiny}   % footnote for bibtex
\setbeamertemplate{bibliography item}[text]   % reference list for bibtex

% 标题页配置项
\title{汽车航迹推算研究进展}
\author{钱隆}
\institute{武汉大学测绘遥感信息工程国家重点实验室}
\date{\today}
\titlegraphic{
    % https://www.whu.edu.cn/xxgk/wdbs.htm 武大标识 01-校徽\02-png\1.1-标准校徽.png
    \includegraphics[width=0.2\linewidth]{pic/Wuhan_University_Logo.png} 
}

\usepackage{Whu}

% defs
\def\cmd#1{\texttt{\color{red}\footnotesize $\backslash$#1}}
\def\env#1{\texttt{\color{blue}\footnotesize #1}}
\definecolor{deepblue}{rgb}{0, 0.1451, 0.3294}
\definecolor{deepred}{rgb}{0.6,0,0}
\definecolor{deepgreen}{rgb}{0.0667, 0.3412, 0.2510}
\definecolor{halfgray}{gray}{0.55}

\lstset{
    basicstyle=\ttfamily\small,
    keywordstyle=\bfseries\color{deepblue},
    emphstyle=\ttfamily\color{deepred},    % Custom highlighting style
    stringstyle=\color{deepgreen},
    numbers=left,
    numberstyle=\small\color{halfgray},
    rulesepcolor=\color{red!20!green!20!blue!20},
    frame=shadowbox}



\begin{document}

\kaishu

% 幻灯片标题页
\begin{frame} % https://tex.stackexchange.com/questions/504002/what-is-a-frame-in-beamer
    \titlepage

\end{frame}

\begin{frame}
    \tableofcontents[sectionstyle=show,subsectionstyle=show/shaded/hide,subsubsectionstyle=show/shaded/hide]
\end{frame}


\section{研究内容}

\subsection{文献}

\begin{frame}{国内外研究现状}
    根据深度神经网络在航迹推算中
    \begin{itemize}
        \item 基于滤波的方法,深度神经网络
        \item 端到端
    \end{itemize}
\end{frame}

\begin{frame}{国内外研究现状}
    
    % https://changhao-chen.github.io/
    % http://deepio.cs.ox.ac.uk/
    \begin{table}
    \tiny
        \begin{tabular}{cccc}
            \toprule
            作者 & 文章题目 & 期刊/会议 & 发表日期 \\
            \midrule
            Changhao Chen, et al & Learning Selective\cite{chen2022learning} & IEEE Trans Neural Netw Learn Syst & 2022 \\
            Ze Chen, et al & Contrastive Learning\cite{chen2022contrastive} & IEEE Sens. J. & 2022 \\
            Changhao Chen, et al & DynaNet\cite{chen2021dynanet} & IEEE Trans Neural Netw Learn Syst & 2021 \\
            Changhao Chen, et al & Deep Neural Network\cite{chen2021deep} & IEEE Trans Mob Comput & 2021 \\
            Changhao Chen, et al & Deep Learning\cite{chen2020deep} & IEEE Internet Things J. & 2020 \\
            Changhao Chen, et al & MotionTransformer\cite{chen2018transferring} & AAAI & 2019 \\
            Changhao Chen, et al & IONet\cite{chen2018ionet} & AAAI & 2018 \\
            \bottomrule
        \end{tabular}
    \end{table}
\end{frame}

\begin{frame}{国内外研究现状}
    
    % https://changhao-chen.github.io/
    % http://deepio.cs.ox.ac.uk/
    \begin{table}
    \tiny
        \begin{tabular}{cccc}
            \toprule
            作者 & 文章题目 & 期刊/会议 & 发表日期 \\
            \midrule
            Zhou Hang, et al & IMU Dead-Reckoning Localization\cite{zhou2022imu} & IROS & 2022 \\
            Brossard Martin, et al & AI-IMU Dead-Reckoning\cite{brossard2020ai} & IEEE Trans Intell. Vehicles & 2020 \\
            \bottomrule
        \end{tabular}
    \end{table}
\end{frame}

\subsection{试验}

\begin{frame}{数据集}
    KITTI数据集IMU采样率为100Hz, L1/L2 RTK下, 参考真值标称位置精度为 \qty{0.01}{\meter}, 速度精度(RMS)为 \qty{0.05}{\km\per\hour}(开阔环境下)
    \begin{itemize}
        \item 车行
        \begin{itemize}
            \item \textbf{Awesome GINS Dataset}\cite{Tang2022}.
            \item Canadian Adverse Driving Conditions Dataset\cite{pitropov2021canadian}.
            \item \textbf{KITTI Dataset}\cite{Geiger2013IJRR}.
        \end{itemize}
        \item 人行
        \begin{itemize}
            \item \textbf{RoNIN Dataset}\cite{herath2020ronin}.
        \end{itemize}   
    \end{itemize}
\end{frame}

\begin{frame}{深度神经网络模型}
    作者使用 Visual Odometry / SLAM Evaluation 2012 子数据集做训练和评估
    \begin{itemize}
        \item \textbf{AI-IMU Dead-Reckoning}\cite{brossard2020ai}.
    \end{itemize}
\end{frame}

\begin{frame}
    KITTI数据集中2011\_09\_30\_drive\_0028采集段用AI-IMU Dead-Reckoning模型预测的结果
    \begin{figure}[htbp]
        \centering
        \includegraphics[scale=0.2]{pic/experiment_kitti_2011_09_30_drive_0028.png}
        \caption{AI-IMU Dead-Reckoning模型预测坐标结果}
    \end{figure}
\end{frame}

\begin{frame}
     KITTI数据集中2011\_09\_30\_drive\_0028采集段初始状态旋转90用AI-IMU Dead-Reckoning模型预测的结果。 除了试验不同的初值,还试验了重采样,初值的选择和相同采样率下的重采样,对航迹推算影响较小
    \begin{figure}[htbp]
        \centering
        \includegraphics[scale=0.2]{pic/experiment_kitti_2011_09_30_drive_0028_init_rotate.png}
        \caption{AI-IMU Dead-Reckoning模型预测坐标结果}
    \end{figure}
\end{frame}

\begin{frame}
    KITTI数据集中2022\_03\_15\_drive\_0001用AI-IMU Dead-Reckoning模型预测失败。可能的原因: 相较于KITTI数据集, 轮式机器人和机动车运动特征差距较大, 手机传感器精度较低, 手机传感器坐标系不一致。
   \begin{figure}[htbp]
       \centering
       \includegraphics[scale=0.2]{pic/experiment_custom_2022_03_15_drive_0001.png}
       \caption{AI-IMU Dead-Reckoning模型预测坐标结果}
   \end{figure}
\end{frame}

\begin{frame}{KITTI数据集IMU传感器参数和手机传感器参数差异}
    \begin{table}
        \tiny
        \begin{tabular}{cccc}
            \toprule
            传感器 & 精度等级 & 陀螺仪零偏稳定性 & 加速度计零偏稳定性\\
            \midrule
            % https://www.oxts.com/products/rt3000-v3/
            % https://www.oxts.com/wp-content/uploads/2021/07/RT3000-v3-datasheet-210702.pdf
            OxTS RT3000 v3             & 战术级   & \qty{2}{\degree\per\hour}       & \qty{2}{\ug} \\
            % https://www.st.com/zh/mems-and-sensors/lsm6dsl.html
            % https://www.st.com/resource/en/datasheet/lsm6dsl.pdf
            STMicroelectronics LSM6DSL & 微机械级 & \qty{\pm 3}{\degree\per\second} & \qty{\pm 40}{\mg} \\
            \bottomrule
        \end{tabular}
    \end{table}
\end{frame}

\section{研究计划}

\begin{frame}
    \begin{enumerate}
        \item 数据集和模型间的交叉验证
        \begin{itemize}
            \item 基于 ai-imu-dr 模型开源代码重新训练滤波模型(作者提供的预训练模型无法加载)
            \item 基于论文\cite{dugne2021understanding}训练端到端模型
            \item 实采数据以验证基于手机传感器通用性
        \end{itemize}

        \item 训练参考真值系统支持
        \begin{itemize}
            \item 尝试基于手机GNSS传感器的PPS信号, 实现同参考真值系统的无线同步
        \end{itemize}
    \end{enumerate}
\end{frame}

\begin{frame}{可能的创新点}
    \begin{enumerate}
        \item 自采数据包含停车场环境倒车轨迹
        \item 模型中包含自适应网络结构识别停车等行为重置参数
    \end{enumerate}
\end{frame}


\begin{frame}[allowframebreaks]{参考文献} % https://stackoverflow.com/questions/913966/how-to-split-table-of-contents-across-multiple-slides-with-latex-beamer
	\printbibliography[heading=references]
\end{frame}


\begin{frame}
    \begin{center}
        {\Huge 谢谢}
    \end{center}
\end{frame}

\end{document}