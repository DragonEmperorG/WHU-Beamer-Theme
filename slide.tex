% https://yuxtech.club/tex/lshort-zh-cn.pdf
% https://tug.ctan.org/macros/latex/contrib/beamer/doc/beameruserguide.pdf
% \documentclass[⟨options⟩]{⟨class-name⟩}
\documentclass{beamer} % 幻灯格式的文档类,使用无衬线字体。

%% 字体
% ctex – LATEX classes and packages for Chinese typesetting
% https://ftp.jaist.ac.jp/pub/CTAN/language/chinese/ctex/ctex.pdf
%\usepackage[⟨options⟩]{⟨package-name⟩}
\usepackage{ctex} % 添加中文及版式的支持

%% 字体
% fontenc – Standard package for selecting font encodings
\usepackage[T1]{fontenc} % 切换字体编码

%% 字体 | 颜色
% xcolor – Driver-independent color extensions for LATEX and pdfLATEX
% https://ftp.yz.yamagata-u.ac.jp/pub/CTAN/macros/latex/contrib/xcolor/xcolor.pdf
\usepackage{xcolor} % 切换字体颜色

%% 段落 | 代码
% listings – Typeset source code listings using LATEX
% https://ftp.jaist.ac.jp/pub/CTAN/macros/latex/contrib/listings/listings.pdf
\usepackage{listings} % 提供了排版关键字高亮的代码环境 lstlisting 以及对版式的自定义。类似宏包有 minted



%% 表格
% booktabs – Publication quality tables in LATEX
\usepackage{booktabs} % 排版三线表

%% 表格
% tabularx – Tabulars with adjustable-width columns
% https://ftp.yz.yamagata-u.ac.jp/pub/CTAN/macros/latex/required/tools/tabularx.pdf
\usepackage{tabularx} % 

%% 插图 | 图片
% graphicx – Enhanced support for graphics
\usepackage{graphicx} % 支持插图

%% 插图
% pgf – Create PostScript and PDF graphics in TEX
\usepackage{tikz} % 支持绘图


%% 符号 | 公式
% amsmath – AMS mathematical facilities for LATEX
% https://ftp.yz.yamagata-u.ac.jp/pub/CTAN/macros/latex/required/amsmath/amsldoc.pdf
\usepackage{amsmath} % AMS 数学公式扩展

%% 符号 | 公式
% siunitx – A comprehensive (SI) units package
% https://ftp.yz.yamagata-u.ac.jp/pub/CTAN/macros/latex/contrib/siunitx/siunitx.pdf
\usepackage{siunitx} % 单位符号

%% 符号 | 符号
% latexsym - The LaTeX symbol fonts for use with LaTeX2ε.
% http://mirrors.ibiblio.org/CTAN/macros/latex/base/latexsym.pdf
\usepackage{latexsym} % 带有角标 ℓ 的符号命令依赖 latexsym 宏包。


%% 布局 | 栏
% multicol – Intermix single and multiple columns
% https://ftp.yz.yamagata-u.ac.jp/pub/CTAN/macros/latex/required/tools/multicol.pdf
\usepackage{multicol} % 提供将内容自由分栏的 multicols 环境


%% 引用
% BibLATEX – Sophisticated Bibliographies in LATEX
% https://ftp.kddilabs.jp/CTAN/macros/latex/contrib/biblatex/doc/biblatex.pdf
% backend=bibtex, bibtex8, biber default: biber
% Specifies the database backend. The following backends are supported:
\usepackage[backend=bibtex,sorting=none]{biblatex}
\addbibresource{ref.bib} % 为 biblatex 引入参考文献数据库
\defbibheading{references}{\section*{参考文献}}   % heading for bibtex

%% 引用
% hyperref – Extensive support for hypertext in LATEX
% https://ftp.yz.yamagata-u.ac.jp/pub/CTAN/macros/latex/contrib/hyperref/doc/hyperref-doc.pdf
% 为减少可能的冲突,习惯上将 hyperref 宏包放在其它宏包之后调用。
\usepackage{hyperref} % hyperref 宏包涉及到的链接遍布 LATEX 的每一个角落——目录、引用、脚注、索引、参考文献等等都被封装成超链接。

\setbeamerfont{footnote}{size=\tiny}   % footnote for bibtex
\setbeamertemplate{bibliography item}[text]   % reference list for bibtex

% 标题页配置项
\title{汽车航迹推算研究进展}
\author{钱隆}
\institute{武汉大学测绘遥感信息工程国家重点实验室}
\date{\today}
\titlegraphic{
    % https://www.whu.edu.cn/xxgk/wdbs.htm 武大标识 01-校徽\02-png\1.1-标准校徽.png
    \includegraphics[width=0.2\linewidth]{pic/Wuhan_University_Logo.png} 
}

\usepackage{Whu}

% defs
\def\cmd#1{\texttt{\color{red}\footnotesize $\backslash$#1}}
\def\env#1{\texttt{\color{blue}\footnotesize #1}}
\definecolor{deepblue}{rgb}{0, 0.1451, 0.3294}
\definecolor{deepred}{rgb}{0.6,0,0}
\definecolor{deepgreen}{rgb}{0.0667, 0.3412, 0.2510}
\definecolor{halfgray}{gray}{0.55}

\lstset{
    basicstyle=\ttfamily\small,
    keywordstyle=\bfseries\color{deepblue},
    emphstyle=\ttfamily\color{deepred},    % Custom highlighting style
    stringstyle=\color{deepgreen},
    numbers=none,
    numberstyle=\small\color{halfgray},
    rulesepcolor=\color{red!20!green!20!blue!20},
    frame=shadowbox
}


\begin{document}

\kaishu

% 幻灯片标题页
\begin{frame} % https://tex.stackexchange.com/questions/504002/what-is-a-frame-in-beamer
    \titlepage
\end{frame}

\begin{frame}
    \tableofcontents[sectionstyle=show,subsectionstyle=show/shaded/hide,subsubsectionstyle=show/shaded/hide]
\end{frame}


\section{研究背景}

\begin{frame}
    研究出发点
    \begin{enumerate}
        \item 高可用高精度室内外无缝定位的需求
        \item 智能手机终端泛用性
    \end{enumerate}
    研究难点
    \begin{enumerate}
        \item 室内环境复杂
        \item 智能手机传感器精度较差
    \end{enumerate}
\end{frame}

\begin{frame}{博士大论文章节安排}
    博士大论文拟定的题目为《基于智能手机的数据和模型双驱动室内定位方法研究》,目前第3和第5章有发表于《基于数据与模型双驱动的音频/惯性传感器耦合定位方法》做支撑,但是作为大论文需要对文章中
    \begin{description}
        \item[1] 绪论
        \item[2] 定位基本原理
        \item[3] 数据驱动行人航迹推算
        \item[4] 数据驱动车载航迹推算
        \item[5] 数据驱动行人航迹推算和音频融合定位算法研究
        \item[6] 数据驱动车载航迹推算和蓝牙融合定位算法研究
        \item[7] 总结和展望
    \end{description}
\end{frame}

\begin{frame}{现有研究成果}
    目前第3和第5章有发表于《测绘学报》的《基于数据与模型双驱动的音频/惯性传感器耦合定位方法》做支撑,但是作为大论文需要对文章中数据驱动行人航迹推算部分模型进行小创新的研究,因为那篇文章中我仅使用了牛老师组提供的模型。
    
    从2022年12月到2023年5月,约6个月的时间对第4章的内容进行研究(在22年大约有零散的2个月对该方向进行了研究),目前使用自采数据,部分验证了基于不变卡尔曼滤波解决行车航迹推算的问题。

\end{frame}

\section{研究内容}

\begin{frame}{数据集}
    \begin{itemize}
        \item 公开数据集
        \begin{itemize}
            \item KITTI (2013)
        \end{itemize}

        \item 自采数据集
        \begin{itemize}
            \item 重庆202304数据集
        \end{itemize}
    \end{itemize}
\end{frame}

\begin{frame}{智能手机(Android)采样率}
	
	% https://www.st.com/resource/en/datasheet/lsm6dsm.pdf
	
	% https://www.bosch-sensortec.com/media/boschsensortec/downloads/datasheets/bst-bmi160-ds000.pdf
	% https://www.st.com/resource/en/datasheet/lis2mdl.pdf
    \begin{table}
    \tiny
        \begin{tabular}{lccc}
            \toprule
	            传感器/手机 (Hz) & HUAWEI Mate30 & HUAWEI P20 & GOOGLE Pixel3 \\
	            \midrule
	            TYPE\_ACCELEROMETER & 500 & 500 & 393(400) \\
				TYPE\_ACCELEROMETER\_UNCALIBRATED & 250 & 250 & 393(400) \\
				TYPE\_GYROSCOPE & 500 & 500 & 393(400) \\
				TYPE\_GYROSCOPE\_UNCALIBRATED & 500 & 500 & 393(400) \\
				TYPE\_MAGNETIC\_FIELD & 100 & 100 & 99(100) \\
				TYPE\_MAGNETIC\_FIELD\_UNCALIBRATED & 100 & 100 & 99(100) \\
				TYPE\_GAME\_ROTATION\_VECTOR & 100 & 100 & 179 \\
				GNSS\_LOCATION & 1 & 1 & 1 \\
				GNSS\_RAW\_MEASURENMENT & 1 & 1 & 1 \\
            \bottomrule
        \end{tabular}
    \end{table}
\end{frame}

\begin{frame}{模型}
    \begin{itemize}
        \item 数据预处理
        \begin{itemize}
            \item 原始数据时间同步
            \item 原始数据转换训练数据
        \end{itemize}

        \item 不变卡尔曼滤波航迹推算模型
        \begin{itemize}
            \item Matlab 分析制图代码
            \item Python 训练代码
        \end{itemize}
    \end{itemize}
\end{frame}

\section{研究计划}

\begin{frame}[allowframebreaks]
    \begin{enumerate}
        \item 完成重庆数据集后处理
        \begin{itemize}
            \item 编写自动化处理脚本
            \item 确认基于 GNSS 传感器时间同步方案有缺陷
        \end{itemize}

        \item 改进基于深度学习自适应不变卡尔曼滤波航迹推算方法
        \begin{itemize}
            \item 修改优化目标,减少保持时间和距离
            \item 增加对前向速度的约束
            \item 参考《IMU Dead-Reckoning Localization with RNN-IEKF Algorithm》在深度学习部分使用 RNN 模型替换 CNN 模型
        \end{itemize}

        \item 研究基于深度学习端到端航迹推算方法
        \begin{itemize}
            \item 参考《Understanding the Behavior of Data-Driven Inertial Odometry With Kinematics-Mimicking Deep Neural Network》实现端到端方法
            \item 复现该论文在公开数据集的结果
            \item 评估端到端模型在自采数据集的结果
        \end{itemize}

        \item 撰写小论文
        \begin{itemize}
            \item 改进的模型在自采的数据集上有可行的航迹推算结果
        \end{itemize}
    \end{enumerate}
\end{frame}


\begin{frame}[allowframebreaks]{参考文献} % https://stackoverflow.com/questions/913966/how-to-split-table-of-contents-across-multiple-slides-with-latex-beamer
	\printbibliography[heading=references]
\end{frame}


\begin{frame}
    \begin{center}
        {\Huge 谢谢}
    \end{center}
\end{frame}

\end{document}