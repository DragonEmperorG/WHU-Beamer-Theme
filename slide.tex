% https://yuxtech.club/tex/lshort-zh-cn.pdf
% https://tug.ctan.org/macros/latex/contrib/beamer/doc/beameruserguide.pdf
% \documentclass[⟨options⟩]{⟨class-name⟩}
\documentclass{beamer} % 幻灯格式的文档类,使用无衬线字体。

%% 字体
% ctex – LATEX classes and packages for Chinese typesetting
% https://ftp.jaist.ac.jp/pub/CTAN/language/chinese/ctex/ctex.pdf
%\usepackage[⟨options⟩]{⟨package-name⟩}
\usepackage{ctex} % 添加中文及版式的支持

%% 字体
% fontenc – Standard package for selecting font encodings
\usepackage[T1]{fontenc} % 切换字体编码

%% 字体 | 颜色
% xcolor – Driver-independent color extensions for LATEX and pdfLATEX
% https://ftp.yz.yamagata-u.ac.jp/pub/CTAN/macros/latex/contrib/xcolor/xcolor.pdf
\usepackage{xcolor} % 切换字体颜色

%% 段落 | 代码
% listings – Typeset source code listings using LATEX
% https://ftp.jaist.ac.jp/pub/CTAN/macros/latex/contrib/listings/listings.pdf
\usepackage{listings} % 提供了排版关键字高亮的代码环境 lstlisting 以及对版式的自定义。类似宏包有 minted



%% 表格
% booktabs – Publication quality tables in LATEX
\usepackage{booktabs} % 排版三线表

%% 表格
% tabularx – Tabulars with adjustable-width columns
% https://ftp.yz.yamagata-u.ac.jp/pub/CTAN/macros/latex/required/tools/tabularx.pdf
\usepackage{tabularx} % 

%% 插图 | 图片
% graphicx – Enhanced support for graphics
\usepackage{graphicx} % 支持插图

%% 插图
% pgf – Create PostScript and PDF graphics in TEX
\usepackage{tikz} % 支持绘图


%% 符号 | 公式
% amsmath – AMS mathematical facilities for LATEX
% https://ftp.yz.yamagata-u.ac.jp/pub/CTAN/macros/latex/required/amsmath/amsldoc.pdf
\usepackage{amsmath} % AMS 数学公式扩展

%% 符号 | 公式
% siunitx – A comprehensive (SI) units package
% https://ftp.yz.yamagata-u.ac.jp/pub/CTAN/macros/latex/contrib/siunitx/siunitx.pdf
\usepackage{siunitx} % 单位符号

%% 符号 | 符号
% latexsym - The LaTeX symbol fonts for use with LaTeX2ε.
% http://mirrors.ibiblio.org/CTAN/macros/latex/base/latexsym.pdf
\usepackage{latexsym} % 带有角标 ℓ 的符号命令依赖 latexsym 宏包。


%% 布局 | 栏
% multicol – Intermix single and multiple columns
% https://ftp.yz.yamagata-u.ac.jp/pub/CTAN/macros/latex/required/tools/multicol.pdf
\usepackage{multicol} % 提供将内容自由分栏的 multicols 环境


%% 引用
% BibLATEX – Sophisticated Bibliographies in LATEX
% https://ftp.kddilabs.jp/CTAN/macros/latex/contrib/biblatex/doc/biblatex.pdf
% backend=bibtex, bibtex8, biber default: biber
% Specifies the database backend. The following backends are supported:
\usepackage[backend=bibtex,sorting=none]{biblatex}
\addbibresource{ref.bib} % 为 biblatex 引入参考文献数据库
\defbibheading{references}{\section*{参考文献}}   % heading for bibtex

%% 引用
% hyperref – Extensive support for hypertext in LATEX
% https://ftp.yz.yamagata-u.ac.jp/pub/CTAN/macros/latex/contrib/hyperref/doc/hyperref-doc.pdf
% 为减少可能的冲突,习惯上将 hyperref 宏包放在其它宏包之后调用。
\usepackage{hyperref} % hyperref 宏包涉及到的链接遍布 LATEX 的每一个角落——目录、引用、脚注、索引、参考文献等等都被封装成超链接。

\setbeamerfont{footnote}{size=\tiny}   % footnote for bibtex
\setbeamertemplate{bibliography item}[text]   % reference list for bibtex

% 标题页配置项
\title{汽车航迹推算研究进展}
\author{钱隆}
\institute{武汉大学测绘遥感信息工程国家重点实验室}
\date{\today}
\titlegraphic{
    % https://www.whu.edu.cn/xxgk/wdbs.htm 武大标识 01-校徽\02-png\1.1-标准校徽.png
    \includegraphics[width=0.2\linewidth]{pic/Wuhan_University_Logo.png} 
}

\usepackage{Whu}

% defs
\def\cmd#1{\texttt{\color{red}\footnotesize $\backslash$#1}}
\def\env#1{\texttt{\color{blue}\footnotesize #1}}
\definecolor{deepblue}{rgb}{0, 0.1451, 0.3294}
\definecolor{deepred}{rgb}{0.6,0,0}
\definecolor{deepgreen}{rgb}{0.0667, 0.3412, 0.2510}
\definecolor{halfgray}{gray}{0.55}

\lstset{
    basicstyle=\ttfamily\small,
    keywordstyle=\bfseries\color{deepblue},
    emphstyle=\ttfamily\color{deepred},    % Custom highlighting style
    stringstyle=\color{deepgreen},
    numbers=none,
    numberstyle=\small\color{halfgray},
    rulesepcolor=\color{red!20!green!20!blue!20},
    frame=shadowbox
}


\begin{document}

\kaishu

% 幻灯片标题页
\begin{frame} % https://tex.stackexchange.com/questions/504002/what-is-a-frame-in-beamer
    \titlepage

\end{frame}

\begin{frame}
    \tableofcontents[sectionstyle=show,subsectionstyle=show/shaded/hide,subsubsectionstyle=show/shaded/hide]
\end{frame}


\section{研究内容}

\subsection{Android 时间系统}

\begin{frame}[fragile]{System 类}
    % https://developer.android.google.cn/reference/java/lang/System#currentTimeMillis()
    \begin{lstlisting}[language = Java]
public static long currentTimeMillis();
    \end{lstlisting}
    Returns the current time in milliseconds. Note that while the unit of time of the return value is a millisecond, 
    the granularity of the value depends on the underlying operating system and may be larger. 
    For example, many operating systems measure time in units of tens of milliseconds.\\
    \begin{tabularx}{\textwidth}{l|X}
        \toprule
        \multicolumn{2}{l}{Returns}\\
        \midrule
        long & the difference, measured in milliseconds, between the current time and midnight, January 1, 1970 UTC.\\
        \bottomrule
    \end{tabularx}
\end{frame}

\begin{frame}[allowframebreaks,fragile]{System 类}
    % https://developer.android.google.cn/reference/java/lang/System#nanoTime()
    \begin{lstlisting}[language = Java]
public static long nanoTime ()
    \end{lstlisting}
    \small {
        Returns the current value of the running Java Virtual Machine's high-resolution time source, in nanoseconds.
        
        This method can only be used to measure elapsed time and is not related to any other notion of system or wall-clock time. 
        The value returned represents nanoseconds since some fixed but arbitrary origin time (perhaps in the future, so values may be negative). 
        The same origin is used by all invocations of this method in an instance of a Java virtual machine; 
        other virtual machine instances are likely to use a different origin.
        
        This method provides nanosecond precision, but not necessarily nanosecond resolution (that is, how frequently the value changes) - 
        no guarantees are made except that the resolution is at least as good as that of currentTimeMillis().

        Differences in successive calls that span greater than approximately 292 years (2$^{63}$ nanoseconds) will not correctly compute elapsed time due to numerical overflow.\\
        The values returned by this method become meaningful only when the difference between two such values, 
        obtained within the same instance of a Java virtual machine, is computed.
        
        The value returned by this method does not account for elapsed time during deep sleep. 
        For timekeeping facilities available on Android see SystemClock.
        
    }
    \begin{tabularx}{\textwidth}{l|X}
        \toprule
        \multicolumn{2}{l}{Returns}\\
        \midrule
        long & the current value of the running Java Virtual Machine's high-resolution time source, in nanoseconds.\\
        \bottomrule
    \end{tabularx}
\end{frame}

\begin{frame}[allowframebreaks]{SystemClock 类}
    % https://developer.android.google.cn/reference/android/os/SystemClock
    Core timekeeping facilities.
    
    Three different clocks are available, and they should not be confused:\\
    \begin{itemize}
        \item System.currentTimeMillis() is the standard "wall" clock (time and date) expressing milliseconds since the epoch. 
        The wall clock can be set by the user or the phone network (see setCurrentTimeMillis(long)), 
        so the time may jump backwards or forwards unpredictably. 
        This clock should only be used when correspondence with real-world dates and times is important, 
        such as in a calendar or alarm clock application. 
        Interval or elapsed time measurements should use a different clock. 
        If you are using System.currentTimeMillis(), 
        consider listening to the ACTION\_TIME\_TICK, ACTION\_TIME\_CHANGED and ACTION\_TIMEZONE\_CHANGED Intent broadcasts to find out when the time changes.
        \item uptimeMillis() is counted in milliseconds since the system was booted. 
        This clock stops when the system enters deep sleep (CPU off, display dark, device waiting for external input), 
        but is not affected by clock scaling, idle, or other power saving mechanisms. 
        This is the basis for most interval timing such as Thread.sleep(millls), Object.wait(millis), and System.nanoTime(). 
        This clock is guaranteed to be monotonic, and is suitable for interval timing when the interval does not span device sleep. 
        Most methods that accept a timestamp value currently expect the uptimeMillis() clock.
        \item elapsedRealtime() and elapsedRealtimeNanos() return the time since the system was booted, and include deep sleep. 
        This clock is guaranteed to be monotonic, and continues to tick even when the CPU is in power saving modes, 
        so is the recommend basis for general purpose interval timing.
    \end{itemize}
\end{frame}

\begin{frame}[fragile]{SystemClock 类}
\label{CLASSSystemClockMETHODelapsedRealtimeNanos}
    % https://developer.android.google.cn/reference/android/os/SystemClock#elapsedRealtimeNanos()
    \begin{lstlisting}[language = Java]
public static long elapsedRealtimeNanos ()
    \end{lstlisting}
    Returns nanoseconds since boot, including time spent in sleep.
    \begin{tabularx}{\textwidth}{l|X}
        \toprule
        \multicolumn{2}{l}{Returns}\\
        \midrule
        long & elapsed nanoseconds since boot.\\
        \bottomrule
    \end{tabularx}
\end{frame}

\begin{frame}[fragile]{SensorEvent 类}
    % https://developer.android.google.cn/reference/android/hardware/SensorEvent#timestamp
    \begin{lstlisting}[language = Java]
public long timestamp
    \end{lstlisting}
    The time in nanoseconds at which the event happened. 
    For a given sensor, each new sensor event should be monotonically increasing using the same time base as \hyperlink{CLASSSystemClockMETHODelapsedRealtimeNanos}{SystemClock.elapsedRealtimeNanos()}.
\end{frame}

\begin{frame}[fragile]{GnssClock 类}
\label{CLASSGnssClockMETHODgetTimeNanos}
    % https://developer.android.google.cn/reference/android/location/GnssClock#getTimeNanos()
    \begin{lstlisting}[language = Java]
public long getTimeNanos ()
    \end{lstlisting}
    Gets the GNSS receiver internal hardware clock value in nanoseconds.

    This value is expected to be monotonically increasing while the hardware clock remains powered on. For the case of a hardware clock that is not continuously on, see the getHardwareClockDiscontinuityCount() field. The GPS time can be derived by subtracting the sum of getFullBiasNanos() and getBiasNanos() (when they are available) from this value. Sub-nanosecond accuracy can be provided by means of getBiasNanos().

    The error estimate for this value (if applicable) is getTimeUncertaintyNanos().
\end{frame}

\begin{frame}[fragile]{GnssClock 类}
    % https://developer.android.google.cn/reference/android/location/GnssClock#getFullBiasNanos()
    \begin{lstlisting}[language = Java]
public long getFullBiasNanos ()
    \end{lstlisting}
    Gets the difference between hardware clock (\hyperlink{CLASSGnssClockMETHODgetTimeNanos}{getTimeNanos()}) inside GPS receiver and the true GPS time since 0000Z, January 6, 1980, in nanoseconds.

    This value is available if the receiver has estimated GPS time. If the computed time is for a non-GPS constellation, the time offset of that constellation to GPS has to be applied to fill this value. The value is only available if hasFullBiasNanos() is true.
    
    The error estimate for the sum of this field and getBiasNanos() is getBiasUncertaintyNanos().
    
    The sign of the value is defined by the following equation:
    \begin{small}
    \begin{equation*}
        local\,estimate\,of\,GPS\,time = TimeNanos - (FullBiasNanos + BiasNanos)
    \end{equation*}
    \end{small}
\end{frame}

\begin{frame}[fragile]{GnssClock 类}
    % https://developer.android.google.cn/reference/android/location/GnssClock#getBiasNanos()
    \begin{lstlisting}[language = Java]
public double getBiasNanos ()
    \end{lstlisting}
    Gets the clock's sub-nanosecond bias.

    See the description of how this field is part of converting from hardware clock time, to GPS time, in getFullBiasNanos().

    The error estimate for the sum of this field and getFullBiasNanos() is getBiasUncertaintyNanos().

    The value is only available if hasBiasNanos() is true.
\end{frame}

\begin{frame}[fragile]{GnssClock 类}
    % https://developer.android.google.cn/reference/android/location/GnssClock#getElapsedRealtimeNanos()
    \begin{lstlisting}[language = Java]
public long getElapsedRealtimeNanos ()
    \end{lstlisting}
    Returns the elapsed real-time of this clock since system boot, in nanoseconds.

    The value is only available if hasElapsedRealtimeNanos() is true.
\end{frame}


\subsection{设计}

\begin{frame}{}
    尽可能遍历试验停车场的所有路径。
    \begin{itemize}
        \item 手机型号.
        \item 手机在车里的位置.
        \item 手机在车里的姿态.
        % http://gk.chengdu.gov.cn/govInfoPub/detail.action?id=990785&tn=2
        % 《机动车停车场场地条件和设施标准》
        % (四)室内停车场车场范围内设计行车速度V≤5km/h,室外停车场车场范围内设计行车速度V≤10km/h。
        \item 行车速度.
        \begin{itemize}
            \item 最低速.
            \item \qty{5}{\km\per\hour}(\qty{1.389}{\m\per\second}).
            \item \qty{10}{\km\per\hour}(\qty{2.778}{\m\per\second}).
            \item \qty{15}{\km\per\hour}(\qty{4.167}{\m\per\second}).
        \end{itemize}
        \item 行车路线.
        \begin{itemize}
            \item 直行.
            \item 倒行.
            \item 左转.
            \item 右转.
            \item 上坡.
            \item 下坡.
        \end{itemize}
    \end{itemize}
\end{frame}

\begin{frame}{}
    
\end{frame}




\section{研究计划}

\begin{frame}
    \begin{enumerate}
        \item 数据集和模型间的交叉验证
        \begin{itemize}
            \item 基于 ai-imu-dr 模型开源代码重新训练滤波模型(作者提供的预训练模型无法加载)
            \item 基于论文\cite{dugne2021understanding}训练端到端模型
            \item 实采数据以验证基于手机传感器通用性
        \end{itemize}

        \item 训练参考真值系统支持
        \begin{itemize}
            \item 尝试基于手机GNSS传感器的PPS信号, 实现同参考真值系统的无线同步
            \item 尝试基于手机IMU传感器和参考真值系统传感器动态的初始同步
        \end{itemize}
    \end{enumerate}
\end{frame}

\begin{frame}{可能的创新点}
    \begin{enumerate}
        \item 基于手机低精度传感器的航迹推算
        \item 自采数据包含停车场环境倒车轨迹
        \item 模型中包含自适应网络结构识别停车等行为重置参数
    \end{enumerate}
\end{frame}


\begin{frame}[allowframebreaks]{参考文献} % https://stackoverflow.com/questions/913966/how-to-split-table-of-contents-across-multiple-slides-with-latex-beamer
	\printbibliography[heading=references]
\end{frame}


\begin{frame}
    \begin{center}
        {\Huge 谢谢}
    \end{center}
\end{frame}

\end{document}