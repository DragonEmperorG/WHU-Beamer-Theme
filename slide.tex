% https://yuxtech.club/tex/lshort-zh-cn.pdf
% https://tug.ctan.org/macros/latex/contrib/beamer/doc/beameruserguide.pdf
% \documentclass[⟨options⟩]{⟨class-name⟩}
\documentclass{beamer} % 幻灯格式的文档类,使用无衬线字体。

%% 字体
% ctex – LATEX classes and packages for Chinese typesetting
% https://ftp.jaist.ac.jp/pub/CTAN/language/chinese/ctex/ctex.pdf
%\usepackage[⟨options⟩]{⟨package-name⟩}
\usepackage{ctex} % 添加中文及版式的支持

%% 字体
% fontenc – Standard package for selecting font encodings
\usepackage[T1]{fontenc} % 切换字体编码

%% 字体 | 颜色
% xcolor – Driver-independent color extensions for LATEX and pdfLATEX
% https://ftp.yz.yamagata-u.ac.jp/pub/CTAN/macros/latex/contrib/xcolor/xcolor.pdf
\usepackage{xcolor} % 切换字体颜色

%% 段落 | 代码
% listings – Typeset source code listings using LATEX
% https://ftp.jaist.ac.jp/pub/CTAN/macros/latex/contrib/listings/listings.pdf
\usepackage{listings} % 提供了排版关键字高亮的代码环境 lstlisting 以及对版式的自定义。类似宏包有 minted



%% 表格
% booktabs – Publication quality tables in LATEX
\usepackage{booktabs} % 排版三线表

%% 表格
% tabularx – Tabulars with adjustable-width columns
% https://ftp.yz.yamagata-u.ac.jp/pub/CTAN/macros/latex/required/tools/tabularx.pdf
\usepackage{tabularx} % 

%% 插图 | 图片
% graphicx – Enhanced support for graphics
\usepackage{graphicx} % 支持插图

\usepackage[skip=0pt]{caption}

%% 插图
% pgf – Create PostScript and PDF graphics in TEX
\usepackage{tikz} % 支持绘图


%% 符号 | 公式
% amsmath – AMS mathematical facilities for LATEX
% https://ftp.yz.yamagata-u.ac.jp/pub/CTAN/macros/latex/required/amsmath/amsldoc.pdf
\usepackage{amsmath} % AMS 数学公式扩展

%% 符号 | 公式
% siunitx – A comprehensive (SI) units package
% https://ftp.yz.yamagata-u.ac.jp/pub/CTAN/macros/latex/contrib/siunitx/siunitx.pdf
\usepackage{siunitx} % 单位符号

%% 符号 | 符号
% latexsym - The LaTeX symbol fonts for use with LaTeX2ε.
% http://mirrors.ibiblio.org/CTAN/macros/latex/base/latexsym.pdf
\usepackage{latexsym} % 带有角标 ℓ 的符号命令依赖 latexsym 宏包。


%% 布局 | 栏
% multicol – Intermix single and multiple columns
% https://ftp.yz.yamagata-u.ac.jp/pub/CTAN/macros/latex/required/tools/multicol.pdf
\usepackage{multicol} % 提供将内容自由分栏的 multicols 环境


%% 引用
% BibLATEX – Sophisticated Bibliographies in LATEX
% https://ftp.kddilabs.jp/CTAN/macros/latex/contrib/biblatex/doc/biblatex.pdf
% backend=bibtex, bibtex8, biber default: biber
% Specifies the database backend. The following backends are supported:
\usepackage[backend=bibtex,sorting=none]{biblatex}
\addbibresource{ref.bib} % 为 biblatex 引入参考文献数据库
\defbibheading{references}{\section*{参考文献}}   % heading for bibtex

%% 引用
% hyperref – Extensive support for hypertext in LATEX
% https://ftp.yz.yamagata-u.ac.jp/pub/CTAN/macros/latex/contrib/hyperref/doc/hyperref-doc.pdf
% 为减少可能的冲突,习惯上将 hyperref 宏包放在其它宏包之后调用。
\usepackage{hyperref} % hyperref 宏包涉及到的链接遍布 LATEX 的每一个角落——目录、引用、脚注、索引、参考文献等等都被封装成超链接。

\setbeamerfont{footnote}{size=\tiny}   % footnote for bibtex
\setbeamertemplate{bibliography item}[text]   % reference list for bibtex

% 标题页配置项
\title{汽车航迹推算研究进展}
\author{钱隆}
\institute{武汉大学测绘遥感信息工程国家重点实验室}
\date{\today}
\titlegraphic{
    % https://www.whu.edu.cn/xxgk/wdbs.htm 武大标识 01-校徽\02-png\1.1-标准校徽.png
    \includegraphics[width=0.2\linewidth]{pic/Wuhan_University_Logo.png} 
}

\usepackage{Whu}

% defs
\def\cmd#1{\texttt{\color{red}\footnotesize $\backslash$#1}}
\def\env#1{\texttt{\color{blue}\footnotesize #1}}
\definecolor{deepblue}{rgb}{0, 0.1451, 0.3294}
\definecolor{deepred}{rgb}{0.6,0,0}
\definecolor{deepgreen}{rgb}{0.0667, 0.3412, 0.2510}
\definecolor{halfgray}{gray}{0.55}

\lstset{
    basicstyle=\ttfamily\small,
    keywordstyle=\bfseries\color{deepblue},
    emphstyle=\ttfamily\color{deepred},    % Custom highlighting style
    stringstyle=\color{deepgreen},
    numbers=none,
    numberstyle=\small\color{halfgray},
    rulesepcolor=\color{red!20!green!20!blue!20},
    frame=shadowbox
}


\begin{document}

\kaishu

% 幻灯片标题页
\begin{frame} % https://tex.stackexchange.com/questions/504002/what-is-a-frame-in-beamer
    \titlepage
\end{frame}

\begin{frame}
    \tableofcontents[sectionstyle=show,subsectionstyle=show/shaded/hide,subsubsectionstyle=show/shaded/hide]
\end{frame}


\section{研究进展}

\begin{frame}
    时间 15:09:51.070  15:11:55.070 距离 \qty{283.566}{\meter}
    设计速度 \qty{15}{\km\per\hour} 匀速
    \begin{figure}[htbp]
        \centering
        \includegraphics[width=0.7\textwidth]{pic/Result_2023_04_10_0008_Mate30_1_24801.png}
        \caption{2023\_04\_10\_drive\_0008\_phone\_huawei\_mate30\_1\_24801 结果}
        \label{fig:transformer-arc}
    \end{figure}
\end{frame}

\begin{frame}
    时间 15:09:51.070  15:11:55.070 距离 \qty{283.566}{\meter}
    设计速度 \qty{15}{\km\per\hour} 匀速
    \begin{figure}[htbp]
        \centering
        \includegraphics[width=0.7\textwidth]{pic/Result_2023_04_10_0008_Mate30_1_24801.png}
        \caption{2023\_04\_10\_drive\_0008\_phone\_huawei\_mate30\_1\_24801 结果}
        \label{fig:transformer-arc}
    \end{figure}
\end{frame}

\begin{frame}
    时间 15:09:51.070  15:11:55.070 距离 \qty{283.566}{\meter}
    设计速度 \qty{15}{\km\per\hour} 匀速
    \begin{figure}[htbp]
        \centering
        \includegraphics[width=0.7\textwidth]{pic/Result_2023_04_10_0008_Mate30_1_24801.png}
        \caption{2023\_04\_10\_drive\_0008\_phone\_huawei\_mate30\_1\_24801 结果}
        \label{fig:transformer-arc}
    \end{figure}
\end{frame}

\begin{frame}
    时间 15:10:27.000  15:11:55.070 距离 \qty{281.487}{\meter}
    设计速度 \qty{15}{\km\per\hour} 匀速
    \begin{figure}[htbp]
        \centering
        \includegraphics[width=0.7\textwidth]{pic/Result_2023_04_10_0008_Mate30_7187_24801.png}
        \caption{2023\_04\_10\_drive\_0008\_phone\_huawei\_mate30\_7187\_24801 结果}
        \label{fig:transformer-arc}
    \end{figure}
\end{frame}

\begin{frame}
    时间 15:09:51.590  15:11:54.590 距离 \qty{281.559}{\meter}
    设计速度 \qty{15}{\km\per\hour} 匀速
    \begin{figure}[htbp]
        \centering
        \includegraphics[width=0.7\textwidth]{pic/Result_2023_04_10_0008_Pixel3_1_24601.png}
        \caption{2023\_04\_10\_drive\_0008\_phone\_google\_pixel3\_1\_24601 结果}
        \label{fig:transformer-arc}
    \end{figure}
\end{frame}

\begin{frame}
    时间 15:10:27.000  15:11:54.590 距离 \qty{281.487}{\meter}
    设计速度 \qty{15}{\km\per\hour} 匀速
    \begin{figure}[htbp]
        \centering
        \includegraphics[width=0.7\textwidth]{pic/Result_2023_04_10_0008_Pixel3_7083_24601.png}
        \caption{2023\_04\_10\_drive\_0008\_phone\_google\_pixel3\_7083\_24601 结果}
        \label{fig:transformer-arc}
    \end{figure}
\end{frame}



\begin{frame}
    时间 15:30:45.080  15:36:01.080 距离 \qty{844.222}{\meter}
    设计速度 \qty{15}{\km\per\hour} 匀速
    \begin{figure}[htbp]
        \centering
        \includegraphics[width=0.7\textwidth]{pic/Result_2023_04_10_0014_Mate30_1_63201.png}
        \caption{2023\_04\_10\_drive\_0014\_phone\_huawei\_mate30\_1\_63201 结果}
        \label{fig:transformer-arc}
    \end{figure}
\end{frame}

\begin{frame}
    时间 15:31:17.000  15:36:01.080 距离 \qty{844.125}{\meter}
    设计速度 \qty{15}{\km\per\hour} 匀速
    \begin{figure}[htbp]
        \centering
        \includegraphics[width=0.7\textwidth]{pic/Result_2023_04_10_0014_Mate30_6385_63201.png}
        \caption{2023\_04\_10\_drive\_0014\_phone\_huawei\_mate30\_6385\_63201 结果}
        \label{fig:transformer-arc}
    \end{figure}
\end{frame}

\section{研究计划}

\begin{frame}[allowframebreaks]
    \begin{enumerate}

        \item 基于重庆数据集后处理结果撰写大论文数据预处理部分
        
        \item 深入分析基于深度学习自适应不变卡尔曼滤波航迹推算方法的结果,解决轨迹不平滑的问题
        \begin{itemize}
            \item 裁切较短轨迹,缩短轨迹评价间隔,试验解决轨迹不平滑的问题
            \item 选取静止轨迹,试验解决静止状态漂移的问题
            \item 深入分析 20240413 易江路 长条环形训练无效的问题  
        \end{itemize}

        \item 改进基于深度学习自适应不变卡尔曼滤波航迹推算方法
        \begin{itemize}
            \item 修改优化目标,减少保持时间和距离
            \item 增加对前向速度的约束
            \item 参考《IMU Dead-Reckoning Localization with RNN-IEKF Algorithm》在深度学习部分使用 RNN 模型替换 CNN 模型
        \end{itemize}

        \item 研究基于深度学习端到端航迹推算方法

        \item 撰写小论文

    \end{enumerate}
\end{frame}


\begin{frame}[allowframebreaks]{参考文献} % https://stackoverflow.com/questions/913966/how-to-split-table-of-contents-across-multiple-slides-with-latex-beamer
	\printbibliography[heading=references]
\end{frame}


\begin{frame}
    \begin{center}
        {\Huge 谢谢}
    \end{center}
\end{frame}

\end{document}