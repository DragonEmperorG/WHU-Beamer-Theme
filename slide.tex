% https://yuxtech.club/tex/lshort-zh-cn.pdf
% https://tug.ctan.org/macros/latex/contrib/beamer/doc/beameruserguide.pdf
% \documentclass[⟨options⟩]{⟨class-name⟩}
\documentclass{beamer} % 幻灯格式的文档类,使用无衬线字体。

%% 字体
% ctex – LATEX classes and packages for Chinese typesetting
% https://ftp.jaist.ac.jp/pub/CTAN/language/chinese/ctex/ctex.pdf
%\usepackage[⟨options⟩]{⟨package-name⟩}
\usepackage{ctex} % 添加中文及版式的支持

%% 字体
% fontenc – Standard package for selecting font encodings
\usepackage[T1]{fontenc} % 切换字体编码

%% 字体 | 颜色
% xcolor – Driver-independent color extensions for LATEX and pdfLATEX
% https://ftp.yz.yamagata-u.ac.jp/pub/CTAN/macros/latex/contrib/xcolor/xcolor.pdf
\usepackage{xcolor} % 切换字体颜色

%% 段落 | 代码
% listings – Typeset source code listings using LATEX
% https://ftp.jaist.ac.jp/pub/CTAN/macros/latex/contrib/listings/listings.pdf
\usepackage{listings} % 提供了排版关键字高亮的代码环境 lstlisting 以及对版式的自定义。类似宏包有 minted



%% 表格
% booktabs – Publication quality tables in LATEX
\usepackage{booktabs} % 排版三线表

%% 表格
% tabularx – Tabulars with adjustable-width columns
% https://ftp.yz.yamagata-u.ac.jp/pub/CTAN/macros/latex/required/tools/tabularx.pdf
\usepackage{tabularx} % 

%% 插图 | 图片
% graphicx – Enhanced support for graphics
\usepackage{graphicx} % 支持插图

\usepackage[skip=0pt]{caption}

%% 插图
% pgf – Create PostScript and PDF graphics in TEX
\usepackage{tikz} % 支持绘图


%% 符号 | 公式
% amsmath – AMS mathematical facilities for LATEX
% https://ftp.yz.yamagata-u.ac.jp/pub/CTAN/macros/latex/required/amsmath/amsldoc.pdf
\usepackage{amsmath} % AMS 数学公式扩展

%% 符号 | 公式
% siunitx – A comprehensive (SI) units package
% https://ftp.yz.yamagata-u.ac.jp/pub/CTAN/macros/latex/contrib/siunitx/siunitx.pdf
\usepackage{siunitx} % 单位符号

%% 符号 | 符号
% latexsym - The LaTeX symbol fonts for use with LaTeX2ε.
% http://mirrors.ibiblio.org/CTAN/macros/latex/base/latexsym.pdf
\usepackage{latexsym} % 带有角标 ℓ 的符号命令依赖 latexsym 宏包。


%% 布局 | 栏
% multicol – Intermix single and multiple columns
% https://ftp.yz.yamagata-u.ac.jp/pub/CTAN/macros/latex/required/tools/multicol.pdf
\usepackage{multicol} % 提供将内容自由分栏的 multicols 环境


%% 引用
% BibLATEX – Sophisticated Bibliographies in LATEX
% https://ftp.kddilabs.jp/CTAN/macros/latex/contrib/biblatex/doc/biblatex.pdf
% backend=bibtex, bibtex8, biber default: biber
% Specifies the database backend. The following backends are supported:
\usepackage[backend=bibtex,sorting=none]{biblatex}
\addbibresource{ref.bib} % 为 biblatex 引入参考文献数据库
\defbibheading{references}{\section*{参考文献}}   % heading for bibtex

%% 引用
% hyperref – Extensive support for hypertext in LATEX
% https://ftp.yz.yamagata-u.ac.jp/pub/CTAN/macros/latex/contrib/hyperref/doc/hyperref-doc.pdf
% 为减少可能的冲突,习惯上将 hyperref 宏包放在其它宏包之后调用。
\usepackage{hyperref} % hyperref 宏包涉及到的链接遍布 LATEX 的每一个角落——目录、引用、脚注、索引、参考文献等等都被封装成超链接。

\setbeamerfont{footnote}{size=\tiny}   % footnote for bibtex
\setbeamertemplate{bibliography item}[text]   % reference list for bibtex

% 标题页配置项
\title{汽车航迹推算研究进展}
\author{钱隆}
\institute{武汉大学测绘遥感信息工程国家重点实验室}
\date{\today}
\titlegraphic{
    % https://www.whu.edu.cn/xxgk/wdbs.htm 武大标识 01-校徽\02-png\1.1-标准校徽.png
    \includegraphics[width=0.2\linewidth]{pic/Wuhan_University_Logo.png} 
}

\usepackage{Whu}

% defs
\def\cmd#1{\texttt{\color{red}\footnotesize $\backslash$#1}}
\def\env#1{\texttt{\color{blue}\footnotesize #1}}
\definecolor{deepblue}{rgb}{0, 0.1451, 0.3294}
\definecolor{deepred}{rgb}{0.6,0,0}
\definecolor{deepgreen}{rgb}{0.0667, 0.3412, 0.2510}
\definecolor{halfgray}{gray}{0.55}

\lstset{
    basicstyle=\ttfamily\small,
    keywordstyle=\bfseries\color{deepblue},
    emphstyle=\ttfamily\color{deepred},    % Custom highlighting style
    stringstyle=\color{deepgreen},
    numbers=none,
    numberstyle=\small\color{halfgray},
    rulesepcolor=\color{red!20!green!20!blue!20},
    frame=shadowbox
}


\begin{document}

\kaishu

% 幻灯片标题页
\begin{frame} % https://tex.stackexchange.com/questions/504002/what-is-a-frame-in-beamer
    \titlepage
\end{frame}

\begin{frame}
    \tableofcontents[sectionstyle=show,subsectionstyle=show/shaded/hide,subsubsectionstyle=show/shaded/hide]
\end{frame}


\section{研究进展}

\subsection{AI-IMU Dead-Reckoning}

\begin{frame}
    AI-IMU Dead-Reckoning \footfullcite{brossard2020ai} 在给定初值的条件下,仅使用6轴IMU,在KITTI数据集上获得和采用LiDAR和双目相机排名靠前方法类似的性能。
\end{frame}

\begin{frame}{系统模型}
    不变扩展卡尔曼滤波状态量
    \begin{align*} \mathbf {x}_n:=\left(\mathbf {R}_n^{\textsc {imu}},\;\mathbf {v}_n^{\textsc {imu}},\;\mathbf {p}_n^{\textsc {imu}},\;\mathbf {b}_{n}^{\boldsymbol{\omega }}, \;\mathbf {b}_{n}^{\mathbf {a}},\; \mathbf {R}_n^{\mathsf {c}}, \;\mathbf {p}_n^{\mathsf {c}}\right) \end{align*}
    其中$ \mathbf {R}_n^{\textsc {imu}}\in SO(3) $,$ \mathbf {v}_n^{\textsc {imu}}\in \mathbb{R}^3 $,$ \mathbf {p}_n^{\textsc {imu}}\in \mathbb{R}^3 $
    ,$ \mathbf {b}_{n}^{\boldsymbol{\omega }} \in \mathbb{R}^3 $,$ \mathbf {b}_{n}^{\mathbf {a}} \in \mathbb{R}^3 $
    ,$ \mathbf {R}_n^{\mathsf {c}} \in SO(3) $,$ \mathbf {p}_n^{\mathsf {c}} \in \mathbb{R}^3 $
    \begin{figure}[htbp]
        \centering
        \includegraphics[width=0.7\textwidth]{pic/bross4-2980758-large.png}
        \caption{AI-IMU Dead-Reckoning 系统模型架构}
        \label{fig:transformer-arc}
    \end{figure}
\end{frame}

\begin{frame}{状态方程}    
    \begin{align*} 
        \mathbf {R}_{n+1}^{\textsc {imu}} &= \mathbf {R}_n^{\textsc {imu}} \exp \left((\boldsymbol{\omega }_n dt)_{\times} \right), \\
        \mathbf {v}_{n+1}^{\textsc {imu}} &= \mathbf {v}_n^{\textsc {imu}} + \left(\mathbf {R}_n^{\textsc {imu}}\mathbf {a}_n + \mathbf {g}\right)dt, \\
        \mathbf {p}_{n+1}^{\textsc {imu}} &= \mathbf {p}_n^{\textsc {imu}} + \mathbf {v}_n^{\textsc {imu}} dt, \\
        \mathbf {b}_{n+1}^{\boldsymbol{\omega }} &= \mathbf {b}_{n}^{\boldsymbol{\omega }} + \mathbf {w}^{\mathbf {b}_{\omega }}_{n}, \\ 
        \mathbf {b}_{n+1}^{\mathbf {a}} &= \mathbf {b}_{n}^{\mathbf {a}} + \mathbf {w}^{\mathbf {b}_{\mathbf {a}}}_{n}, \\
        \mathbf {R}_{n+1}^{\mathsf {c}} &=\mathbf {R}_n^{\mathsf {c}}\exp ((\mathbf {w}_n^{\mathbf {R}^{\mathsf {c}}})_{\times}), \\
        \mathbf {p}_{n+1}^{\mathsf {c}} &=\mathbf {p}_n^{\mathsf {c}}+ \mathbf {w}_n^{\mathbf {p}^{\mathsf {c}}}. 
    \end{align*}
\end{frame}

\begin{frame}{观测方程}
    \begin{align*} 
        \boldsymbol{\omega }_n^{\textsc {imu}} &= \boldsymbol{\omega }_n + \mathbf {b}_{n}^{\boldsymbol{\omega }} + \mathbf {w}_{n}^{\boldsymbol{\omega }}, \\ 
        \mathbf {a}^{\textsc {imu}}_n &= \mathbf {a}_n + \mathbf {b}_{n}^{\mathbf {a}} + \mathbf {w}_{n}^{\mathbf {a}}, \\
        \mathbf {y}_{n} &= \begin{bmatrix}y_{n}^{\mathrm{lat}} \\ y_{n}^{\mathrm{up}} \end{bmatrix} = \begin{bmatrix}h^{\mathrm{lat}}(\mathbf {x}_{n})+\mathrm{n}_{n}^{\mathrm{lat}} \\ h^{\mathrm{up}}(\mathbf {x}_{n})+\mathrm{n}_{n}^{\mathrm{up}} \end{bmatrix} = \begin{bmatrix}v_n^{\mathrm{lat}} \\ v_n^{\mathrm{up}} \end{bmatrix}+ \mathbf{n}_n.
    \end{align*}
    $ \mathbf{n}_n=\begin{bmatrix}n_n^{\mathrm{lat}},n_n^{\mathrm{up}}\end{bmatrix} $,观测噪声协方差矩阵$ \mathbf{N}_n\in\mathbb{R}^{2\times2} $
\end{frame}

\begin{frame}{基于人工智能的测量误差自适应调节}
    \begin{figure}[htbp]
        \centering
        \includegraphics[width=0.6\textwidth]{pic/MesNet.png}
        \caption{网络模型}
    \end{figure}
\end{frame}

\begin{frame}{评价指标}
    \begin{enumerate}
        \item Relative Translation Error
        \item Relative Rotational Error
    \end{enumerate}
\end{frame}

\begin{frame}{KITTI Visual Odometry / SLAM Evaluation 2012 子数据集}
    子数据集中提供真值的轨迹总长度约为\qty{22.2}{\km},总时长约为\qty{40}{\minute}
    \begin{table}
        \tiny
        \begin{tabular}{ccccc}
            \toprule
            序列号 & 轨迹名称 & 长度(\unit{\km}) & 时长(\unit{\s}) & 类别 \\
            \midrule
            00 & 2011\_10\_03\_drive\_0027 & 3.7 & 471 & Residential \\
            01 & 2011\_10\_03\_drive\_0042 & 2.5 & 114 & Road \\
            02 & 2011\_10\_03\_drive\_0034 & 5.1 & 483 & Residential \\
            03 & 2011\_09\_26\_drive\_0067 & 0.6 & 83  & Unknown  \\
            04 & 2011\_09\_30\_drive\_0016 & 0.4 & 28  & Road \\
            05 & 2011\_09\_30\_drive\_0018 & 2.2 & 288 & Residential \\
            06 & 2011\_09\_30\_drive\_0020 & 1.2 & 114 & Residential \\
            07 & 2011\_09\_30\_drive\_0027 & 0.7 & 114 & Residential \\
            08 & 2011\_09\_30\_drive\_0028 & 3.2 & 423 & Residential \\
            09 & 2011\_09\_30\_drive\_0033 & 1.7 & 165 & Residential \\
            10 & 2011\_09\_30\_drive\_0034 & 0.9 & 124 & Residential \\
            \bottomrule
        \end{tabular}
    \end{table}
\end{frame}

\begin{frame}{统计结果}
    \begin{figure}[htbp]
        \centering
        \includegraphics[width=1\textwidth]{pic/bross.t1-2980758-large.png}
    \end{figure}
\end{frame}

\begin{frame}{试验结果轨迹01}
    \begin{figure}[htbp]
        \centering
        \includegraphics[width=0.8\textwidth]{pic/bross7-2980758-large.png}
    \end{figure}
\end{frame}

\begin{frame}{试验结果轨迹07}
    \begin{figure}[htbp]
        \centering
        \includegraphics[width=0.7\textwidth]{pic/bross5-2980758-large.png}
    \end{figure}
\end{frame}

\begin{frame}{试验结果轨迹08}
    \begin{figure}[htbp]
        \centering
        \includegraphics[width=0.6\textwidth]{pic/bross1-2980758-large.png}
    \end{figure}
\end{frame}

\begin{frame}{试验结果轨迹09}
    \begin{figure}[htbp]
        \centering
        \includegraphics[width=0.6\textwidth]{pic/bross6-2980758-large.png}
    \end{figure}
\end{frame}

\begin{frame}{2023年4月10日自采数据集华为Mate30}
    数据集中提供真值的轨迹总长度约为\qty{7.8}{\km},总时长约为\qty{62}{\minute},包含不同速度,不同轨迹,包括停车,再启动,倒车。
    \begin{table}
        \scriptsize
        \begin{tabular}{cccc}
            \toprule
            轨迹名称 & 长度(\unit{\m}) & 时长(\unit{\s}) & 类别 \\
            \midrule
            2023\_10\_03\_drive\_0008 & 282 & 124 & 矩形 \\
            2023\_10\_03\_drive\_0009 & 285 & 122 & 矩形 \\
            2023\_10\_03\_drive\_0010 & 285 & 208 & 矩形 \\
            2023\_10\_03\_drive\_0011 & 520 & 168 & S形 \\
            2023\_10\_03\_drive\_0012 & 525 & 216 & S形 \\
            2023\_10\_03\_drive\_0013 & 524 & 370 & S形 \\
            2023\_10\_03\_drive\_0014 & 844 & 316 & 矩形 \\
            2023\_10\_03\_drive\_0015 & 1556 & 529 & S形 \\
            2023\_10\_03\_drive\_0016 & 593 & 459 & S形 \\
            2023\_10\_03\_drive\_0017 & 839 & 418 & 矩形 \\
            2023\_10\_03\_drive\_0018 & 1553 & 807 & S形 \\
            \bottomrule
        \end{tabular}
    \end{table}
\end{frame}

\begin{frame}{试验场地万达广场(大渡口店)停车场}
    \begin{figure}[htbp]
        \centering
        \includegraphics[width=0.7\textwidth]{pic/万达广场停车场.jpg}
    \end{figure}
\end{frame}

\begin{frame}
    时间 15:09:51.070  15:11:55.070 距离 \qty{283.566}{\meter}
    设计速度 \qty{15}{\km\per\hour} 匀速
    \begin{figure}[htbp]
        \centering
        \includegraphics[width=0.7\textwidth]{pic/Result_2023_04_10_0008_Mate30_1_24801.png}
        \caption{2023\_04\_10\_drive\_0008\_phone\_huawei\_mate30\_1\_24801 结果}
        \label{fig:transformer-arc}
    \end{figure}
\end{frame}

\begin{frame}
    时间 15:10:27.000  15:11:55.070 距离 \qty{281.487}{\meter}
    设计速度 \qty{15}{\km\per\hour} 匀速
    \begin{figure}[htbp]
        \centering
        \includegraphics[width=0.7\textwidth]{pic/Result_2023_04_10_0008_Mate30_7187_24801.png}
        \caption{2023\_04\_10\_drive\_0008\_phone\_huawei\_mate30\_7187\_24801 结果}
        \label{fig:transformer-arc}
    \end{figure}
\end{frame}

\begin{frame}
    时间 15:09:51.590  15:11:54.590 距离 \qty{281.559}{\meter}
    设计速度 \qty{15}{\km\per\hour} 匀速
    \begin{figure}[htbp]
        \centering
        \includegraphics[width=0.7\textwidth]{pic/Result_2023_04_10_0008_Pixel3_1_24601.png}
        \caption{2023\_04\_10\_drive\_0008\_phone\_google\_pixel3\_1\_24601 结果}
        \label{fig:transformer-arc}
    \end{figure}
\end{frame}

\begin{frame}
    时间 15:10:27.000  15:11:54.590 距离 \qty{281.487}{\meter}
    设计速度 \qty{15}{\km\per\hour} 匀速
    \begin{figure}[htbp]
        \centering
        \includegraphics[width=0.7\textwidth]{pic/Result_2023_04_10_0008_Pixel3_7083_24601.png}
        \caption{2023\_04\_10\_drive\_0008\_phone\_google\_pixel3\_7083\_24601 结果}
        \label{fig:transformer-arc}
    \end{figure}
\end{frame}



\begin{frame}
    时间 15:30:45.080  15:36:01.080 距离 \qty{844.222}{\meter}
    设计速度小于 \qty{15}{\km\per\hour} 匀速
    \begin{figure}[htbp]
        \centering
        \includegraphics[width=0.7\textwidth]{pic/Result_2023_04_10_0014_Mate30_1_63201.png}
        \caption{2023\_04\_10\_drive\_0014\_phone\_huawei\_mate30\_1\_63201 结果}
        \label{fig:transformer-arc}
    \end{figure}
\end{frame}

\begin{frame}
    时间 15:31:17.000  15:36:01.080 距离 \qty{844.125}{\meter}
    设计速度 \qty{15}{\km\per\hour} 匀速
    \begin{figure}[htbp]
        \centering
        \includegraphics[width=0.7\textwidth]{pic/Result_2023_04_10_0014_Mate30_6385_63201.png}
        \caption{2023\_04\_10\_drive\_0014\_phone\_huawei\_mate30\_6385\_63201 结果}
        \label{fig:transformer-arc}
    \end{figure}
\end{frame}

\begin{frame}
    时间 15:17:51.005  15:20:33.080 距离 \qty{520.066}{\meter}
    设计速度 \qty{15}{\km\per\hour} 匀速
    \begin{figure}[htbp]
        \centering
        \includegraphics[width=0.7\textwidth]{pic/Result_2023_04_10_0011_Mate30_1_33601.png}
        \caption{2023\_04\_10\_drive\_0011\_phone\_huawei\_mate30\_1\_33601 结果}
        \label{fig:transformer-arc}
    \end{figure}
\end{frame}

\begin{frame}
    时间 15:17:51.005  15:20:33.080 距离 \qty{520.066}{\meter}
    设计速度 \qty{15}{\km\per\hour} 匀速
    \begin{figure}[htbp]
        \centering
        \includegraphics[width=0.7\textwidth]{pic/Result_2023_04_10_0011_Mate30_1_33601_Constrained.png}
        \caption{2023\_04\_10\_drive\_0011\_phone\_huawei\_mate30\_1\_33601 前向速度强约束结果}
        \label{fig:transformer-arc}
    \end{figure}
\end{frame}

\subsection{A Novel Deep Odometry Network for Vehicle Positioning Based on Smartphone}

\begin{frame}
    A Novel Deep Odometry Network for Vehicle Positioning Based on Smartphone \footfullcite{wang2023novel} 在无GNSS信号区域,集成IMU、气压计和DeepOdo实现航迹推算,相较于NHCs辅助的IMU推算方法在水平和垂直方向上的平均绝对误差分别有73.1\%和98.33\%的提升。
\end{frame}

\begin{frame}{系统模型}
    \begin{figure}[htbp]
        \centering
        \includegraphics[width=0.7\textwidth]{pic/weng1-3240227-large.png}
        \caption{系统模型架构}
    \end{figure}    
\end{frame}

\begin{frame}
    \begin{figure}[htbp]
        \centering
        \includegraphics[width=0.3\textwidth]{pic/weng2-3240227-large.png}
        \caption{DeepOdo 网络模型}
    \end{figure}
    损失函数
    \begin{equation*} \text {loss}=\frac {1}{N}\sum _{i=1}^{N} \left ({v_{\mathrm {Odo}}-\hat {v}_{\mathrm {Odo}} }\right)^{2}\end{equation*}
\end{frame}

\begin{frame}
    \begin{figure}[htbp]
        \centering
        \includegraphics[width=0.9\textwidth]{pic/weng9abcdef-3240227-large.png}
        \caption{不同网络模型结果}
    \end{figure}    
\end{frame}

\begin{frame}
    \begin{figure}[htbp]
        \centering
        \includegraphics[width=0.9\textwidth]{pic/weng10-3240227-large.png}
        \caption{不同网络模型误差}
    \end{figure}    
\end{frame}

\begin{frame}{中环及湾仔绕道隧道试验}
    \qty{4}{\km},\qty{207}{\s}
    \begin{figure}[htbp]
        \centering
        \includegraphics[width=1\textwidth]{pic/weng14-3240227-large.png}
        \caption{试验轨迹结果}
    \end{figure}    
\end{frame}

\begin{frame}{中环及湾仔绕道隧道试验}
    MAE \qty{0.87}{\m\per\s}
    \begin{figure}[htbp]
        \centering
        \includegraphics[width=1\textwidth]{pic/weng13-3240227-large.png}
        \caption{试验速度结果}
    \end{figure}    
\end{frame}

\begin{frame}
    \begin{figure}[htbp]
        \centering
        \includegraphics[width=0.9\textwidth]{pic/2023_04_10_0010_Mate30.png}
        \caption{2023\_04\_10\_drive\_0010\_phone\_huawei\_mate30 结果}
    \end{figure}
\end{frame}

\begin{frame}
    \begin{figure}[htbp]
        \centering
        \includegraphics[width=0.9\textwidth]{pic/2023_04_10_0015_Mate30.png}
        \caption{2023\_04\_10\_drive\_0015\_phone\_huawei\_mate30 结果}
    \end{figure}
\end{frame}

\begin{frame}
    \begin{figure}[htbp]
        \centering
        \includegraphics[width=0.9\textwidth]{pic/2023_04_10_0016_Mate30.png}
        \caption{2023\_04\_10\_drive\_0016\_phone\_huawei\_mate30 结果}
    \end{figure}
\end{frame}


\section{研究计划}

\begin{frame}[allowframebreaks]
    \begin{enumerate}
        \item 改进基于深度学习自适应不变卡尔曼滤波航迹推算方法
        \begin{itemize}
            \item 修改优化目标,减少保持时间和距离
            \item 增加对前向速度的约束
            \item 参考《IMU Dead-Reckoning Localization with RNN-IEKF Algorithm》在深度学习部分使用 RNN 模型替换 CNN 模型
        \end{itemize}
    \end{enumerate}
\end{frame}


\begin{frame}[allowframebreaks]{参考文献} % https://stackoverflow.com/questions/913966/how-to-split-table-of-contents-across-multiple-slides-with-latex-beamer
	\printbibliography[heading=references]
\end{frame}


\begin{frame}
    \begin{center}
        {\Huge 谢谢}
    \end{center}
\end{frame}

\end{document}